%chktex-file 36
%chktex-file 8
%chktex-file 11
%chktex-file 24
\documentclass[ComputationalMathematics.tex]{subfiles}

\begin{document}

%%%%%%%%%%%%%%~~~~~~~~~~~~~~~~~~~~~~~~~~~~~~~~~~~~~~~~%%%%%%%%%%%%%%%
\section{26th of October 2018 --- F. Poloni}
%%%%%%%%%%%%%%~~~~~~~~~~~~~~~~~~~~~~~~~~~~~~~~~~~~~~~~%%%%%%%%%%%%%%%

\subsection{How to construct a QR factorization}

In the previous lecture we introduced the $QR$ factorization and we defined what an Householder reflector is.

At the end of the lecture we gave a first MatLab implementation of {\tt householder\_vector}:

\begin{center}
\begin{minipage}{.9\linewidth}
  \begin{algorithm}[H]
    \caption{Householder vector Matlab implementation.}\label{alg:26ott1}
    \begin{minted}[linenos=true]{matlab}
      function[v,s] = householdervector(x)
      s = norm(x);
      v = x;
      v(1) = v(1) - s;
      v = v / norm(v);
    \end{minted}
  \end{algorithm}
\end{minipage}
\end{center}
\vspace{0.5cm}

What's the problem of this algorithm? That the subtraction may create a problem with machine numbers, if $s$ and $\|x\|$ are very close. If we take $\|x\| = - s$ the subtraction becomes and addition, and everything works well.

In the end, we would like to obtain this behaviour for every possible value for $x$ and $s$, so line $2$ may be modified as {\tt s = - sign(x(1)) * norm(x)}.


\addpic{0.5}{pics/26ott/1.png}{If $x$ is oriented as in the plot it's better if we choose $-\|x\| e_{1}$ verse, since it's opposite to $x$.}{fig:26ott1}

\begin{example}
Given $A = \begin{pmatrix}
  \times & \times & \times & \times & \times\\
  \times & \times & \times & \times & \times\\
  \times & \times & \times & \times & \times\\
  \times & \times & \times & \times & \times\\
  \times & \times & \times & \times & \times\\
\end{pmatrix} \in \mathcal{M}(m, m, \R)$, where $m=5$, we would like to calculate the $QR$ factorization of $A$.
\begin{description}
  \item[{\sc Step 1}]: costruct a Householder matrix that sends $A(:, 1)$ (first column of $A$) to a multiple of $e_{1}$. Then we have $H_{1}A = \begin{pmatrix}
  \times & \times & \times & \times & \times\\
  0 & \times & \times & \times & \times\\
  0 & \times & \times & \times & \times\\
  0 & \times & \times & \times & \times\\
  0 & \times & \times & \times & \times\\
  \end{pmatrix}$
 
  \item[{\sc Step 2}]: take $H_{2} \in \mathcal{M}(m-1, m-1, \mathds{R})$ such that $H_{2}A(2:end, 2) = \begin{pmatrix} \times\\ 0\\ 0\\ 0\\ \end{pmatrix}$ and compute:
    \begin{equation}
      \begin{aligned}
        Q_2 (H_1 A)
        &= 
        \begin{pmatrix}
          1 & 0 & 0 & 0 & 0\\
          0 & & &\\
          0 && & \text{\Huge H}_2\\
          0 &&& \\
          0 &&&\\
        \end{pmatrix} 
        \begin{pmatrix}
          \times & \times & \times & \times & \times\\
          0 & \times & \times & \times & \times\\
          0 & \times & \times & \times & \times\\
          0 & \times & \times & \times & \times\\
          0 & \times & \times & \times & \times\\
        \end{pmatrix}\\
        &= \begin{pmatrix}
          \times & \times & \times & \times & \times\\
          0 & \times & \times & \times & \times\\
          0 & 0 & \times & \times & \times\\
          0 & 0 & \times & \times & \times\\
          0 & 0 & \times & \times & \times\\
        \end{pmatrix}
      \end{aligned}
    \end{equation}
    
    And we denote $Q_{2} =\begin{pmatrix}
        I_{1 \times 1} & 0\\
        0 & H_{2}\\
      \end{pmatrix}$, $Q_{1}=H_1$;
  \item[{\sc Step 3}]: take $H_{3} \in \mathcal{M}(m-2, m-2, \mathds{R})$ such that $H_{3}A(3:end, 3) = \begin{pmatrix} \times\\ 0\\ 0\\ \end{pmatrix}$
      and we compute:
      \begin{equation}
        \begin{aligned}
          Q_3 (Q_2 Q_1 A)
          &=  \begin{pmatrix}
          1 & 0 & 0 & 0 & 0\\
          0 & 1 & 0 & 0 & 0\\
          0 & 0 & & \\
          0 & 0 && \text{\huge H}_3\\
          0 & 0 &&\\
        \end{pmatrix} 
        \begin{pmatrix}
          \times & \times & \times & \times & \times\\
          0 & \times & \times & \times & \times\\
          0 & 0 & \times & \times & \times\\
          0 & 0 & \times & \times & \times\\
          0 & 0 & \times & \times & \times\\
        \end{pmatrix}\\
        &= \begin{pmatrix}
          \times & \times & \times & \times & \times\\
          0 & \times & \times & \times & \times\\
          0 & 0 & \times & \times & \times\\
          0 & 0 & 0 & \times & \times\\
          0 & 0 & 0 & \times & \times\\
        \end{pmatrix}
        \end{aligned}
      \end{equation}

      So, $Q_{3} =
      \begin{pmatrix}
        I_{2 \times 2} & 0\\
        0 & H_{3}\\
      \end{pmatrix}$;

  \item[{\sc Step 4}]: take $H_{4} \in \mathcal{M}(m-3, m-3, \mathds{R})$ such that $H_{4}A(4:end, 4) = \begin{pmatrix} \times\\ 0\\ \end{pmatrix}$ and we compute:

   \begin{equation}
        \begin{aligned}
          Q_4 (Q_3 Q_2 Q_1 A)
          &=  \begin{pmatrix}
          1 & 0 & 0 & 0 & 0\\
          0 & 1 & 0 & 0 & 0\\
          0 & 0 & 1 & \\
          0 & 0 && \text{\Large H}_4\\
          0 & 0 &&\\
        \end{pmatrix}
        \begin{pmatrix}
          \times & \times & \times & \times & \times\\
          0 & \times & \times & \times & \times\\
          0 & 0 & \times & \times & \times\\
          0 & 0 & 0 & \times & \times\\
          0 & 0 & 0 & \times & \times\\
        \end{pmatrix}\\
        &= \begin{pmatrix}
          \times & \times & \times & \times & \times\\
          0 & \times & \times & \times & \times\\
          0 & 0 & \times & \times & \times\\
          0 & 0 & 0 & \times & \times\\
          0 & 0 & 0 & 0 & \times\\
        \end{pmatrix}
        \end{aligned}
      \end{equation}

Where, $Q_{4} =
      \begin{pmatrix}
        I_{3 \times 3} & 0\\
        0 & H_{4}\\
      \end{pmatrix}$. 
\end{description}

  In the end, since $Q_{i}$ is an orthogonal matrix and the product of orthogonal matrices is orthogonal, $Q_{1} Q_{2} Q_{3} Q_{4} A = T$, which is an upper triangular matrix.
\end{example}


\begin{theorem}[Product of block matrices]
  Let $I \in \mathcal{M}(k, k, \R)$, let $H_{i} \in \mathcal{M}(m-k, m-k, \R)$ and let $B_{i} \in \mathcal{M}(k, k, \R)$, $C_{i} \in \mathcal{M}(k, m-k, \R)$ and $A_{i} \in \mathcal{M}(m-k, m-k, \R)$, then the product between the two following block matrices is exactly the one showed below.

\[\begin{pmatrix}
    I & 0\\
    0 & H_{i}\\
  \end{pmatrix} 
  \begin{pmatrix}
    B_{i} & C_{i}\\
    0 & A_{i}\\
  \end{pmatrix}
  = 
  \begin{pmatrix}
    B_{i}I & C_{i}\\
    0 & H_{i} A_{i}\\
  \end{pmatrix}
\]
\end{theorem}

\begin{proof}
  It's trivial computation, using the definition of matrix product.
\end{proof}

\subsubsection{Matlab implementation}

\begin{center}
\begin{minipage}{.9\linewidth}
\begin{algorithm}[H]
\begin{minted}[linenos=true]{matlab}
function [Q, R] = myqr(A)
[m, n] = size(A);
Q = eye(m);
for j = 1:n
    v = householder_vector(A(j:end, j));
    H = eye(length(v)) - 2*v*v';
    A(j:end,j:end) = H * A(j:end,j:end);
    Q(:, j:end) = Q(:, j:end) * H;
end
R = A;
\end{minted}
  \label{alg:26ottQR1}
\caption{First implementation of QR factorization.}
\end{algorithm}
\end{minipage}
\end{center}

\begin{proposition}
  The cost of this implementation when $A$ is a square matrix is $O(n^{3} + {(n-1)}^{3} + \cdots + 1^{3})$. If $A$ is a rectangular matrix, then the computational complexity is $O(m \cdot n^{2} + {(m-1) \cdot (n-1)}^{2} + \cdots + {(m - n + 1)}^{3})$.
\end{proposition}

\begin{proof}
  Line $7$ does a matrix product between matrices of size $n$, $n-1$, \ldots, $1$, so the resulting cost is  $O(m \cdot n^{2} + {(m-1) \cdot (n-1)}^{2} + \cdots + {(m - n + 1)}^{3})$.

\end{proof}

We may design a faster algorithm, since $HA_j = A_j - 2v(v^T A_j)$.
\begin{center}
\begin{minipage}{.9\linewidth}
\begin{algorithm}[H]
\begin{minted}[linenos=true]{matlab}
function [Q, A] = myqr(A)
[m, n] = size(A);
Q = eye(m);
for j = 1:n-1
    [v, s] = householder_vector(A(j:end, j));
    A(j,j) = s; A(j+1:end,j) = 0;
    A(j:end,j+1:end) = A(j:end,j+1:end) - ...
        2*v*(v'*A(j:end,j+1:end));
    Q(:, j:end) = Q(:, j:end) - Q(:,j:end)*v*2*v';
end
\end{minted}
  \label{alg:26ottQR2}
\caption{More efficient implementation of QR factorization.}
\end{algorithm}
\end{minipage}
\end{center}

Let's suppose that $A$ is square matrix, partitioned as:

\[A = \begin{pmatrix}
    &\\
  A_{1} & A_{2}\\
  &\\
\end{pmatrix} =
 \begin{pmatrix}
   &\\
   Q_{1} & Q_{2}\\
   &\\
\end{pmatrix}
\begin{pmatrix}
  R_{1, 1} & R_{1,2}\\
  0 & R_{2,2}
  \end{pmatrix}
\]

Then we can recover the factorization of $A_{1}$ from the factorization of $A$, since $A_{1} = Q \cdot R_{11}$.

\begin{proposition}[Thin QR factorization] 
  We may replace $Q \in \mathcal{M}(m, m, \R)$ and $R \in \mathcal{M}(m, m, \R)$ with $Q_1 \in \mathcal{M}(m, n, \R)$ and $R_1 \in \mathcal{M}(n, n, \R)$ and the same factorization holds: $A=QR=Q_1R_1$. This is called \textbf{thin QR factorization}.
\end{proposition}
\begin{proof}
 $A_{1} \in \mathcal{M}(m, n, \R)$, $ A =\begin{pmatrix}
    &\\
    Q_{1} & Q_{2}\\
    &\\
  \end{pmatrix}
  \cdot
  \begin{pmatrix}
    R_{1}\\
    0
  \end{pmatrix}
  =
  Q R
  =
    Q_{1} R_{1}
  +
    Q_{2} 0
  =
  Q_{1} R_{1}$
\end{proof}
In order to save space we may work in the following way:

$Q B =
  \begin{pmatrix}
    1 & \times &\cdots & \times\\
    0 &&&&\\
    \vdots &&\text{\Large I-2V}_1 \tr{\text{\Large V}_1}&\\
    0 &&&&\\
    0 &&&&
  \end{pmatrix} 
  \begin{pmatrix}
    1 & \times & \times & \cdots & \times\\
    0 & 1 &\times & \cdots& \times\\
    \vdots & 0 &&\text{\Large I-2V}_2 \tr{\text{\Large V}_2}\\
    0 & \vdots &&&\\
    0 & 0 &&
  \end{pmatrix} \ldots \\ \ldots
  \begin{pmatrix}
    1 & \times & \times & \cdots & \times\\
    0 & 1 & \times & \cdots & \times\\
    \vdots & 0 & 1 &\\
    0 & \vdots & 0 &&\text{\Large I-2V}_n \tr{\text{\Large V}_n}\\
    0 & 0 & 0 &
  \end{pmatrix} B$.

\begin{myframe}{\bf Fun fact}
There are some libraries that store the $v_{i}$ vectors in the lower part of matrix $R$ which is upper triangular and has only zeros below the main diagonal.
\end{myframe}

\subsection{How to use the thin QR factorization to solve a least squares problem}

We would like to solve $\norm{Ax-b}$ $\forall A \in \mathcal{M}(m, n, \R)$ and $\forall B \in \R^n$ where $m > n$ (aka $A$ is a tall, thin matrix),through the $QR$ factorization.
We would like to solve $\min \norm{Ax-b}$ through the $QR$ factorization.

We may write first the $QR$ factorization of $A$, so $\forall A \in \mathcal{M}(m, n, \R)$, $\exists Q \in \mathcal{M}(m, m, \R)$, $\exists R \in \mathcal{M}(m, n, \R)$ such that $A = QR$, where 
$Q = \begin{pmatrix}
  &\\
  Q_1 & Q_2\\
  &\\
\end{pmatrix}$ and $R = \begin{pmatrix}
    R_1\\0
\end{pmatrix}$.

Then, 
\begin{equation}
\begin{aligned}
  \norm{A\x-\bb} &= \norm{\tr{Q}(A\x-\bb)} = \norm{\tr{Q}QR\x - \tr{Q}\bb} \\
  & = \norm{R\x-\tr{Q}\bb} = \norm{\begin{pmatrix}
    R_1\\0
\end{pmatrix}\x - \begin{pmatrix}
  \tr{Q_{1}}\\
    \tr{Q_{2}}
\end{pmatrix}\bb}\\
&=\norm{\begin{pmatrix}
  R_{1} \x - \tr{Q_{1}} \bb\\
    \tr{Q_{2}}\bb
\end{pmatrix}}
\end{aligned}
\end{equation}
How can we pick $\x$ to minimize the norm of $A\x - \bb$?

Can we choose $x$ such that $R_{1} \x - \tr{Q_{1}} \bb = 0$? Yes, we can, since this is a linear square system, so $\x = \inv{R_{1}} \tr{Q_{1}} \bb $

$\norm{A\x - \bb} = \norm{\tr{Q_{2}} \bb}$


We used the fact that $R_{1}$ is invertible, but is it always true that $R_{1}$ is invertible?

\begin{lemma}
  $R_{1}$ is invertible $\Leftrightarrow$ $A$ has full column rank.
\end{lemma}

\begin{proof}
  $A$ has full column rank $\Leftrightarrow$ $\tr{A}A$ is positive definite $\Leftrightarrow$ $\tr{A}A$ is positive semidefinite and invertible, but $\tr{A}A$ is positive semidefinite, so we only need to prove its invertibility.
  
  Let's compute $\tr{QR} QR = \tr{R} \tr{Q} Q R = \tr{R} R = \begin{pmatrix} \tr{R_{1}} & 0 \end{pmatrix} \begin{pmatrix} R_{1}\\ 0 \end{pmatrix} = \tr{R_{1}} R_1$.
    
    So, $\tr{A}A$ is invertible $\Leftrightarrow$ $R_1$ is invertible.
\end{proof}

\begin{myframe}{\bf Note}
$\tr{R_{1}}R_{1}$ is the Cholesky factorization of $\tr{A}A$.
\end{myframe}

The computational complexity is asymptotically equal to the one of computing the $QR$ factorization, since the other operations are cheaper (the product $Q_1^Tb$ costs $O(mn)$ and solving the triangular linear system by back-substitution costs $O(n^2)$).
\end{document}


