\documentclass[preview,12pt]{article}
\usepackage[utf8]{inputenc}
\usepackage[english]{babel}
\usepackage[section]{algorithm}
\usepackage{mdframed,xcolor}

\usepackage{algpseudocode}
\usepackage{amsfonts}
\usepackage{amsmath}

\usepackage{amssymb}
\usepackage{amsthm}
\usepackage{array}
\usepackage{attachfile}

\usepackage{bbding}
\usepackage{booktabs}
\usepackage{breqn}

\usepackage{cancel}
\usepackage[font=small, labelfont=sc]{caption}
\usepackage{chngcntr}
\usepackage{cleveref}
\usepackage{color,listings}

\usepackage{dsfont}

\usepackage{eurosym}

\counterwithin{figure}{section}
\usepackage{fixltx2e}
\usepackage{float}
\usepackage{fontawesome}

\usepackage[T1]{fontenc}
\usepackage{fontspec}

\usepackage{frcursive}
\usepackage{fullpage}

\usepackage{graphicx}

%\usepackage[hidelinks]{hyperref}

\usepackage{lmodern}

\usepackage{makeidx}
\usepackage{mathrsfs}
%\usepackage{mathpazo}
\usepackage{mathtools}
\usepackage{microtype}
\usepackage{minted}
\usepackage{multicol}
\usepackage{multirow}
\usepackage{mymacros}

%\usepackage{nccmath}

\usepackage{pifont}

\usepackage{relsize}

\usepackage{subfiles}
\usepackage{subfig}

\usepackage{tableof}
\usepackage{tabto}
\usepackage{tabularx}
\usepackage[many]{tcolorbox}
\usepackage{tikz}
\setlength{\marginparwidth}{2cm}
\usepackage{todonotes}

\usepackage{verbatim}
\usepackage{vwcol}  

\usepackage{xparse} % Add support for \NewDocumentEnvironment
%\usepackage[framemethod=tikz]{mdframed,xcolor}

%%%%%%%MUST STAY HERE%%%%%%%%%%%
%%%%%%%%%%%%%%%%%%%%%%%%%%%%%%%%

\newtcolorbox{myframe}[2][]{
  width=\linewidth-1pt,
  boxrule=0.3pt,
  fonttitle=\bfseries,
  colback=black!5!white,
  colframe=gray,
  title={#2},
  #1
}%

\newtcolorbox{myrecap}[2][]{
  width=\linewidth-1pt,
  boxrule=0.3pt,
  fonttitle=\bfseries,
  colback=blue!5!white,
  colframe=blue,
  title={#2},
  #1
}%

\newtcolorbox{mysyntax}[2][]{
  width=\linewidth-1pt,
  boxrule=0.3pt,
  fonttitle=\bfseries,
  colback=red!5!white,
  colframe=red,
  title={#2},
  #1
}%

\definecolor{bg}{rgb}{0.95,0.95,0.95}

\addto\captionsitalian{\renewcommand{\appendixname}{Allegato}}

\hypersetup{
  pdfborder = {0 0 0}
}
\captionsetup[algorithm]{font=small, labelfont=sc}

%%%%%%%%%%%%%%%%%%%%%%%%%%%%%%%%%%%%%%%%%%%%%%%%%%%%%%%%%%%%%%%%%%%%%%%%%%%%%%%%%%

\begin{document}

\begin{titlepage}
\begin{center}
\vspace{3cm}

\Large

\vspace{2cm}

\includegraphics[scale=0.3]{pics/Cherubino.jpg}

\vspace{2.5cm}

{\Huge \sc Computational Mathematics}

\vspace{2cm}
Based on prof. Antonio Frangioni's and\\
  prof. Federico Poloni's lectures

\vspace{2cm}
First draft provided by Gemma Martini
\vfill

\today

\end{center}
\end{titlepage}

%%%%%%%%%%%%%%%%%%%~~~~~~~~~~~~~~~~~~~~~~~~~~~~~~~~~~~~~~~~~~~%%%%%%%%%%%%%%%%%%%%%%

\tableofcontents
\let\tableofcontents\relax

\newpage

\subfile{19settembre}

\newpage

\subfile{20settembre}

\newpage

\subfile{21settembre}

\newpage

\subfile{26settembre}

\newpage

\subfile{27settembre}

\newpage

%\subfile{28settembre}


%%%%%%%%%%%%%%%%%%%%%%%%%~~~~~~~~~~~~~%%%%%%%%%%%%%%%%%%%%%%%%%%%%%
\newpage

%\subfile{3ottobre}

\newpage

%\subfile{4ottobre}

\newpage

%\subfile{5ottobre}

\newpage

%\subfile{10ottobre}

\newpage

%\subfile{11ottobre}

\newpage

%\subfile{17ottobre}

\newpage

\subfile{18ottobre}

\newpage

\subfile{19ottobre}

\newpage

\subfile{24ottobre}

\newpage

\subfile{25ottobre}

\newpage

\subfile{26ottobre}

\newpage
%%%%%%%%%%%%%%%%%%%%%%%%%~~~~~~~~~~~~~%%%%%%%%%%%%%%%%%%%%%%%%%%%%%

\subfile{7novembre}

\newpage

\subfile{8novembre}

\newpage

\subfile{9novembre}

\newpage

\subfile{14novembre}

\newpage

\subfile{15novembre}

\newpage

\subfile{16novembre}

\newpage

\subfile{21novembre}

\newpage

\subfile{22novembre}

\newpage

\subfile{23novembre}

\newpage

\subfile{28novembre}

\newpage

\subfile{29novembre}

\newpage

\subfile{30novembre}

\newpage
%%%%%%%%%%%%%%%%%%%%%%%%%~~~~~~~~~~~~~%%%%%%%%%%%%%%%%%%%%%%%%%%%%%

\subfile{5dicembre}

\newpage

\subfile{6dicembre}

\newpage

\subfile{7dicembre}

\newpage

\subfile{12dicembre}

\newpage

\subfile{13dicembre}

\newpage

\subfile{14dicembre}

\end{document}


